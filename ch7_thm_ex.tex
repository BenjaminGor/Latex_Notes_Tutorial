\chapter{Framed Theorems and Exercises}

\paragraph{Introduction} This chapter touches on how to make beautiful, colored frames around theorems, examples, and so on. The typesetting and management of exercises and answers will also be discussed. 

\section{Colored Boxes for Theorems and Proofs}

\paragraph{Colored Boxes by (new)tcolorbox}
We will start by generating a simple colored box first, which is most easily done by importing the \texttt{tcolorbox} package. Then we can define the design of the box by the \texttt{\textbackslash newtcolorbox} command. An illustrative template is
\begin{lstlisting}
\newtcolorbox{mybox}[1][]{
  colback=Green!20, 
  colframe=Gray,
  coltitle=Yellow,
  title=This is my box,
  boxrule=1pt,
  leftrule=1ex,
  boxsep=1ex,
  left=1ex,
  right=1ex,
  sharp corners,
  breakable,
  before skip=\topsep,
  after skip=\topsep, #1}
\end{lstlisting}
that creates a box environment named \texttt{mybox}. Then typing
\begin{lstlisting}
\begin{mybox}
The content goes here.
\end{mybox}    
\end{lstlisting}
produces the following box:
\begin{mybox}
The content goes here.
\end{mybox}
\texttt{colback}, \texttt{colframe}, and \texttt{coltitle} denote the color of the background, frame, and title correspondingly. \texttt{boxrule} (\texttt{leftrule}) is the width of bounding lines (on the left), and \texttt{boxsep} indicates the overall padding around the title and content. \texttt{left}/\texttt{right} further refines the padding to the left/right. The meanings of the \texttt{sharp corners} and \texttt{breakable} keywords are not hard to guess: the box will have sharp corners instead of rounded ones, and it can break across pages. Finally, \texttt{before skip} and \texttt{after skip} indicate the vertical spacing to other objects before and after the entire box. The \texttt{[1][]} part is added so that the box can receive an optional argument, which can override the given setting of the box via putting \texttt{\#1} at the end of the \texttt{\textbackslash newtcolorbox} option list. 

\paragraph{Colored Numbered Theorems/Examples by newtcbtheorem}
The readers are probably concerned more about how to construct a colored, numbered box for theorems, examples, definitions, and the like. This requires us to pass
\begin{lstlisting}
\tcbuselibrary{theorems}
\end{lstlisting}
and then we can use the \texttt{\textbackslash newtcbtheorem[<init>]\{<name>\}\{<display\_\allowbreak name>\}\{<options>\}\{<prefix>\}} construct. The \texttt{init} part sets up the way of numbering, while the \texttt{options} part is just like the previous input lists for \texttt{\textbackslash newtcolorbox}. For example, we can define
\begin{lstlisting}
\newtcbtheorem[number within=chapter]{thm}{Theorem}{
  colback=Green!20,
  colframe=Green!50,
  fonttitle=\bfseries,
  boxrule=1pt,
  boxsep=1ex,
  left=1ex,
  right=1ex,
  pad after break=1.5ex,
  sharp corners,
  breakable,
  before skip=\topsep,
  after skip=\topsep}{thm}
\end{lstlisting}
where we have added some new parameters: \texttt{fonttitle} here indicates the title to be in boldface, and \texttt{pad after break} adds some padding after the box breaks across the page. The \texttt{number within=chapter} option tells the numbering to be based on chapters, and it can be changed to, e.g.\ \texttt{section}. Subsequently, writing
\begin{lstlisting}
\begin{thm}{Mean Value Theorem}{mvt}
If ...
\begin{equation}
f'(c) = \frac{f(b)-f(a)}{b-a}
\end{equation}
\end{thm}
\end{lstlisting}
produces
\begin{thm}{Mean Value Theorem}{mvt}
If $f(x)$ is continuous on $[a,b]$ and differentiable on $(a,b)$, then there exists $c \in (a,b)$ such that
\begin{equation}
f'(c) = \frac{f(b)-f(a)}{b-a}
\end{equation}
\end{thm}
We can refer to this theorem as Theorem \ref{thm:mvt} by \texttt{\textbackslash ref\{thm:mvt\}} (\texttt{prefix:alias}). It is also possible to supply additional options to override the base setting of the colored box.

\paragraph{Shared Numbering for Definitions and the Others}
Another feature that may be useful is to enable shared numbering for definitions, lemmas, corollaries, properties, and so on. To do so, we can invoke the \texttt{\textbackslash newtcbtheorem} command again and pass the keyword \texttt{use counter from=} for the \texttt{init} option:
\begin{lstlisting}
\tcbset{common/.style={
  colback=Green!20,
  colframe=Green!50,
  fonttitle=\bfseries,
  coltitle=black,
  theorem style=plain,
  ...}
}
\newtcbtheorem[use counter from=thm]{defn}{Definition}{common}{defn}
\end{lstlisting}
Here we use \texttt{\textbackslash tcbset} to save the style for repeated use. Then
\begin{lstlisting}
\begin{defn}{Taylor Expansion}{taylor}
For a function $f(x)$ infinitely differentiable at point $x = a$, we have its Taylor series as
\begin{equation}
f(x) = f(a) + \frac{f'(a)}{1!}(x-a) + \frac{f''(a)}{2!}(x-a)^2 + \frac{f'''(a)}{3!}(x-a)^3 + \cdots 
\end{equation}
\end{defn}
\end{lstlisting}
is rendered as
\begin{defn}{Taylor Expansion}{taylor}
For a function $f(x)$ infinitely differentiable at point $x = a$, we have its Taylor series as
\begin{equation}
f(x) = f(a) + \frac{f'(a)}{1!}(x-a) + \frac{f''(a)}{2!}(x-a)^2 + \frac{f'''(a)}{3!}(x-a)^3 + \cdots 
\end{equation}
\end{defn}
Notice that we have set \texttt{theorem style=plain} (there are more possible values, like \texttt{break}), see if you can figure out where the difference is.

\paragraph{Proofs with tcolorboxenvironment} For proofs or worked steps, their setting will often be slightly different. We can load the \texttt{amsthm} package, which comes along with the \texttt{proof} environment, and apply \texttt{\textbackslash tcolorboxenvironment} on it. The template will be
\begin{lstlisting}
\tcolorboxenvironment{proof}{
  blank,
  breakable,
  borderline west={0.5ex}{2pt}{black},
  left=2ex,
  before skip=\topsep,
  after skip=\topsep} 
\end{lstlisting}
The \texttt{borderline west} option draws a long, thin black line to the left. Then, writing
\begin{lstlisting}
\begin{proof}
Consider $\vec{w} = \vec{u} + t\vec{v}$, ...
\begin{align*}
\Delta = b^2 - 4ac &\leq 0 \\
(2(\vec{u} \cdot \vec{v}))^2 - 4\norm{\vec{u}}^2\norm{\vec{v}}^2 &\leq 0 \\
(\vec{u} \cdot \vec{v})^2 - \norm{\vec{u}}^2\norm{\vec{v}}^2 &\leq 0 \\
(\vec{u} \cdot \vec{v})^2 &\leq \norm{\vec{u}}^2\norm{\vec{v}}^2 \\
|\vec{u} \cdot \vec{v}| &\leq \norm{\vec{u}}\norm{\vec{v}} \qedhere % putting \qedhere to eliminate the spurious gap
\end{align*}
\end{proof}
\end{lstlisting}
results in
\begin{proof}
Consider $\vec{w} = \vec{u} + t\vec{v}$, where $t$ is any scalar, then $\norm{\vec{w}}^2 = \vec{w}\cdot\vec{w} \geq 0$ by positivity. Also, $\vec{w}\cdot\vec{w}$ can be written as a quadratic polynomial in $t$:
\begin{align*}
\vec{w}\cdot\vec{w} = (\vec{u} + t\vec{v}) \cdot (\vec{u} + t\vec{v}) = \norm{\vec{u}}^2 + 2t(\vec{u} \cdot \vec{v}) + t^2\norm{\vec{v}}^2
\end{align*}
Since this quantity is always greater than or equal to zero, i.e.\ the quadratic polynomial has no root or a repeated root, it means that the discriminant must be negative or zero. So,
\begin{align*}
\Delta = b^2 - 4ac &\leq 0 \\
(2(\vec{u} \cdot \vec{v}))^2 - 4\norm{\vec{u}}^2\norm{\vec{v}}^2 &\leq 0 \\
(\vec{u} \cdot \vec{v})^2 - \norm{\vec{u}}^2\norm{\vec{v}}^2 &\leq 0 \\
(\vec{u} \cdot \vec{v})^2 &\leq \norm{\vec{u}}^2\norm{\vec{v}}^2 \\
|\vec{u} \cdot \vec{v}| &\leq \norm{\vec{u}}\norm{\vec{v}} \qedhere
\end{align*}
\end{proof}
Moreover, we can copy it with "Proof" replaced by "Solution" as 
\begin{lstlisting}
\newenvironment{solution}{\begin{proof}[Solution]}{\end{proof}}
\end{lstlisting}

\begin{exercisebox}
\begin{Exercise}
\phantomsection%
\label{exer:colorbox}%
Design your own color box to display any example problem, followed by a worked solution.
\end{Exercise}
\end{exercisebox}

\section{Typesetting Exercises and Answers}

\paragraph{Exercises}
The essence of any math book is its exercises. To deliver exercises and their solutions, we can import the \texttt{exercise} package with the following options (to be explained soon):
\begin{lstlisting}
\usepackage[lastexercise,answerdelayed]{exercise}
\end{lstlisting}
Then we can typeset any exercise within the \texttt{Exercise} environment, which we have been doing for so long. As an example, the last exercise was created by
\begin{lstlisting}
\begin{exercisebox}
\begin{Exercise}
\phantomsection
\label{exer:colorbox}
Design your own color box to display any example problem, followed by a worked solution.
\end{Exercise}
\end{exercisebox}
\end{lstlisting}
where \texttt{exercisebox} here is a self-defined\footnote{The actual implementation can be checked from my raw source code.} colored box environment generated by \texttt{newtcolorbox} as introduced in the last section. We can refer to this exercise as Exercise \ref{exer:colorbox} via its label \texttt{\textbackslash ref\{exer:colorbox\}}. As an extra note, to ensure the internal referencing link to the exercise is correct, we need a patch by inserting \texttt{\textbackslash phantomsection} at its start before \texttt{label}.

\paragraph{Answers}
To typeset the answer for an exercise, we can use the \texttt{Answer} environment supplied with the corresponding label of that exercise. Here we will take Exercise \ref{exer:modulo} as a demonstration, where we can write
\begin{lstlisting}
\begin{Exercise}
\phantomsection
\label{exer:modulo}
... % the exercise
\end{Exercise}
\begin{Answer}[ref=exer:modulo]
... % the answer goes here
\end{Answer}
\end{lstlisting}
Alternatively, one can omit the \texttt{ref} part if the \texttt{lastexercise} option has been ticked when importing the package. It assumes that the answer is for the latest exercise, and hence we can type it immediately after that exercise.

The \texttt{answerdelayed} option saves all the answers until the end, and we can output all of them for once by \texttt{\textbackslash shipoutAnswer}. You should be able to see the answers for Exercise \ref{exer:modulo} and others at the end of the book. The detailed code for the answer section of this book is
\begin{lstlisting}
\cleardoubleoddpage
\chapter*{Answers to Exercises}
\addcontentsline{toc}{chapter}{Answers to Exercises} % add the answer section to the table of content
\ohead{Answer to Exercises}
\shipoutAnswer
\end{lstlisting}

\paragraph{Headers for Answers} The default headers for answers may be too plain. To customize them, we can use the solution proposed in \href{https://tex.stackexchange.com/questions/369265/math-book-how-to-write-exercise-and-answers}{\TeX{} StackExchange 369265} which involves declaring a boolean variable \texttt{firstanswerofthechapter} by \texttt{\textbackslash newboolean} in the \texttt{ifthen} package. A minimal version is
\begin{lstlisting}
\newboolean{firstanswerofthechapter}  
\renewcommand{\AnswerHeader}{\ifthenelse{\boolean{firstanswerofthechapter}}
    {\textbf{Answers for Chapter \thechapter}\par\vspace{1ex}%
     \theExercise)}
    {\theExercise)}
}
\end{lstlisting}
Then we can update the \texttt{\textbackslash AnswerHeader} command with an if-then-else statement. We can call \texttt{\textbackslash setboolean\{firstanswerofthechapter\}\{true\}} whenever we are at the first answer in a chapter, then \texttt{\textbackslash AnswerHeader} will first print the heading "Answers for Chapter \texttt{\textbackslash thechapter}" where \texttt{\textbackslash thechapter} is the chapter counter, and proceed to print the exercise number stored by \texttt{\textbackslash theExercise} with a round bracket to the right. Afterwards, reset \texttt{\textbackslash setboolean\{first\allowbreak answerofthechapter\}\{false\}} and the \texttt{\textbackslash AnswerHeader} will just consist of the exercise number.