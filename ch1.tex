\chapter{The Basic Set-up and Structure of a \LaTeX{} Book}

\paragraph{Introduction}
The first chapter discusses how to properly configure \LaTeX{} files and organize the content's structure so that we can generate our first readable \LaTeX{} book PDF. 

\section{Class, Options, and Packages}

\paragraph{Class}
For each \LaTeX{} document, we need to specify its \textit{class}. Throughout this book, we will use the \verb|scrbook| class provided by the \textbf{KOMA-Script}. To do so, we write \texttt{\textbackslash documentclass\{scrbook\}} at the very beginning (\textit{preamble}) of the main document. Although not explored in this book, some other notable classes that may be of use include \verb|beamer|, \verb|moderncv|, and \verb|article| (or \verb|scrartcl|).

\paragraph{Options} The \verb|scrbook| class provides several options to customize the layout of the book. We can either supply the arguments when declaring the class, or use the command \texttt{\textbackslash KOMAoptions} in the preamble. In this book, we have used
\begin{lstlisting}
\KOMAoptions{paper=a4, fontsize=12pt, chapterprefix=true, twoside=semi, DIV=classic, parskip=half}
\end{lstlisting}
The options are given inside the curly brackets. Clearly, the \verb|paper| option tells the page to be in A4 size while the \verb|fontsize| indicates the font to be 12 pt large. The remaining of them will be explained as we go through later chapters.

The Battle of Berlin, designated as the Berlin Strategic Offensive Operation by the Soviet Union, and also known as the Fall of Berlin, was one of the last major offensives of the European theatre of World War II.[g]

After the Vistula–Oder Offensive of January–February 1945, the Red Army had temporarily halted on a line 60 km (37 mi) east of Berlin. On 9 March, Germany established its defence plan for the city with Operation Clausewitz. The first defensive preparations at the outskirts of Berlin were made on 20 March, under the newly appointed commander of Army Group Vistula, General Gotthard Heinrici.

When the Soviet offensive resumed on 16 April, two Soviet fronts (army groups) attacked Berlin from the east and south, while a third overran German forces positioned north of Berlin. Before the main battle in Berlin commenced, the Red Army encircled the city after successful battles of the Seelow Heights and Halbe. On 20 April 1945, Hitler's birthday, the 1st Belorussian Front led by Marshal Georgy Zhukov, advancing from the east and north, started shelling Berlin's city centre, while Marshal Ivan Konev's 1st Ukrainian Front broke through Army Group Centre and advanced towards the southern suburbs of Berlin. On 23 April General Helmuth Weidling assumed command of the forces within Berlin. The garrison consisted of several depleted and disorganised Army and Waffen-SS divisions, along with poorly trained Volkssturm and Hitler Youth members. Over the course of the next week, the Red Army gradually took the entire city.

On 30 April, Hitler killed himself. The city's garrison surrendered on 2 May but fighting continued to the north-west, west, and south-west of the city until the end of the war in Europe on 8 May (9 May in the Soviet Union) as some German units fought westward so that they could surrender to the Western Allies rather than to the Soviets.[15]

On 12 January 1945, the Red Army began the Vistula–Oder Offensive across the Narew River and from Warsaw, a three-day operation on a broad front, which incorporated four army Fronts.[16] On the fourth day, the Red Army broke out and started moving west, up to 30 to 40 km (19 to 25 mi) per day, taking East Prussia, Danzig, and Poznań, drawing up on a line 60 km (37 mi) east of Berlin along the Oder River.[17]

The new Army Group Vistula (Reichsführer-SS Heinrich Himmler), conducted Operation Solstice, a counter-attack, but this had failed by 24 February.[18][19] The Red Army then drove on to Pomerania, clearing the right bank of the Oder River, thereby reaching into Silesia.[17]

In the south, Soviet and Romanian forces conducted the Siege of Budapest. Three German divisions' attempts to relieve the city failed, and Budapest fell to the Soviets on 13 February.[20] Adolf Hitler insisted on a counter-attack to recapture the Drau-Danube triangle.[21] The goal was to secure the oil region of Nagykanizsa and regain the Danube River for future operations but the depleted German forces had been given an impossible task.[22][23] By 16 March, the German Operation Spring Awakening (also the Lake Balaton Offensive) had failed, and a counter-attack by the Red Army took back in 24 hours everything the Germans had taken ten days to gain.[24] On 30 March, the Soviets entered Austria; and in the Vienna Offensive they captured Vienna on 13 April.[25]

On 12 April 1945, Hitler, who had earlier decided to remain in the city against the wishes of his advisers, heard the news that the American President Franklin D. Roosevelt had died.[26] This briefly raised false hopes in the Führerbunker that there might yet be a falling out among the Allies and that Berlin would be saved at the last moment, as had happened once before when Berlin was threatened (see the Miracle of the House of Brandenburg).[27]

No plans were made by the Western Allies to seize the city.[28] The Supreme Commander [Western] Allied Expeditionary Force, General Eisenhower, lost interest in the race to Berlin and saw no further need to suffer casualties by attacking a city that would be in the Soviet sphere of influence after the war, envisioning excessive friendly fire if both armies attempted to occupy the city at once.[29][30] The main Western Allied contribution to the battle was the bombing of Berlin during 1945.[31] During 1945 the United States Army Air Forces launched mass day raids on Berlin and for 36 nights in succession, scores of Royal Air Force (RAF) Mosquitos bombed the German capital, ending on the night of 20/21 April 1945 just before the Soviets entered the city.[32]

The Soviet offensive into central Germany, which later became East Germany, had two objectives. Stalin did not believe the Western Allies would hand over territory occupied by them in the post-war Soviet zone, so he began the offensive on a broad front and moved rapidly to meet the Western Allies as far west as possible. But the overriding objective was to capture Berlin.[33] The two goals were complementary because possession of the zone could not be won quickly unless Berlin was taken. Another consideration was that Berlin itself held useful post-war strategic assets, including Adolf Hitler and the German nuclear weapons program[34] (but unknown to the Soviet Union, by the time of the Battle of Berlin, the bulk of the uranium and most of the scientists had been evacuated to Haigerloch in the Black Forest).[35] On 6 March, Hitler appointed Lieutenant General Helmuth Reymann commander of the Berlin Defence Area, replacing Lieutenant General Bruno Ritter von Hauenschild.[36]

On 20 March, General Gotthard Heinrici was appointed Commander-in-Chief of Army Group Vistula, replacing Himmler.[37] Heinrici was one of the best defensive tacticians in the German army, and he immediately started to lay defensive plans. Heinrici correctly assessed that the main Soviet thrust would be made over the Oder River and along the main east-west Autobahn.[38] He decided not to try to defend the banks of the Oder with anything more than a light skirmishing screen. Instead, Heinrici arranged for engineers to fortify the Seelow Heights, which overlooked the Oder River at the point where the Autobahn crossed them.[39] This was some 17 km (11 mi) west of the Oder and 90 km (56 mi) east of Berlin. Heinrici thinned out the line in other areas to increase the manpower available to defend the heights. German engineers turned the Oder's flood plain, already saturated by the spring thaw, into a swamp by releasing the water from a reservoir upstream. Behind the plain on the plateau, the engineers built three belts of defensive emplacements[39] reaching back towards the outskirts of Berlin (the lines nearer to Berlin were called the Wotan position).[40] These lines consisted of anti-tank ditches, anti-tank gun emplacements, and an extensive network of trenches and bunkers.[39][40]

On 9 April, after a long resistance, Königsberg in East Prussia fell to the Red Army. This freed up Marshal Rokossovsky's 2nd Belorussian Front to move west to the east bank of the Oder river.[41] Marshal Georgy Zhukov concentrated his 1st Belorussian Front, which had been deployed along the Oder river from Frankfurt (Oder) in the south to the Baltic, into an area in front of the Seelow Heights.[42] The 2nd Belorussian Front moved into the positions being vacated by the 1st Belorussian Front north of the Seelow Heights. While this redeployment was in progress, gaps were left in the lines; and the remnants of General Dietrich von Saucken's German II Army, which had been bottled up in a pocket near Danzig, managed to escape into the Vistula delta.[43] To the south, Marshal Konev shifted the main weight of the 1st Ukrainian Front out of Upper Silesia and north-west to the Neisse River.[3]

The three Soviet fronts had altogether 2.5 million men (including 78,556 soldiers of the 1st Polish Army), 6,250 tanks, \textbf{7,500} aircraft, 41,600 artillery pieces and mortars, 3,255 truck-mounted Katyusha rocket launchers (nicknamed 'Stalin's Organ'), and 95,383 motor vehicles, many manufactured in the US.[3] \thechapter