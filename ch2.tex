\chapter{Formatting of Text and Paragraphs}

\paragraph{Introduction} This chapter explains how to adjust the various aspects of text, such as fonts, shape/size/style, and positioning.

\section{About Fonts}

\subsection{The Three Font Family Types}

\paragraph{(Sans) Serif, Typewriter} In any \LaTeX{} document, the text can be typed in three different \textit{font families}: \textit{serif}, \textit{sans serif}, and \textit{typewriter}. In this book, headings (of chapters, sections, etc.) are in the sans serif family, while the remaining main text is in serif. Table \ref{tab:fontfamily} below demonstrates how to select a specific font family for a piece of text.
\begin{table}
\begin{tabularx}{\textwidth}{|l|X|l|l|}
\hline
Font Family & Command & Switch & Output \\
\hline
Serif & \texttt{\textbackslash textrm\{Hello World!\}}& \texttt{\textbackslash rmfamily} & \textrm{Hello World!} \\
\hline
Sans Serif & \texttt{\textbackslash textsf\{Hello World!\}}& \texttt{\textbackslash sffamily} & \textsf{Hello World!} \\
\hline
Typewriter & \texttt{\textbackslash texttt\{Hello World!\}}& \texttt{\textbackslash ttfamily} & \texttt{Hello World!} \\
\hline
\end{tabularx}
\caption{The commands for switching between the three font families and how they appear.}
\label{tab:fontfamily}
\end{table}
For instance, both
\begin{lstlisting}
... 
produces the following output: \par
\textsf{\lipsum[3]} % or {\sffamily \lipsum[3]}, remember the curly brackets {} to limit the scope of the \sffamily command.
\end{lstlisting}
produces the following output: \par
{\sffamily \lipsum[3]} \par
The \% symbol indicates a trailing comment that is neither interpreted nor displayed.

\subsection{Changing the Actual Font for a Font Family}

\paragraph{Font Libraries}
Each of the previous font families is internally assigned a specific \textit{font}. To change the actual fonts, we can call the corresponding font package(s). The \LaTeX{} Font Catalogue \href{https://tug.org/FontCatalogue/}{https://tug.org/FontCatalogue/} provides a comprehensive list of available fonts and the way to import them. This book has substituted the \textbf{Noto Sans} font for the sans serif family, via the preamble
\begin{lstlisting}
\usepackage[T1]{fontenc}
\usepackage[sf]{noto}
\end{lstlisting}

\begin{exercisebox}
\begin{Exercise}
Choose a font of your liking from the Font Catalogue to replace the original one in the book.
\end{Exercise}
\end{exercisebox}

\section{Text Attributes}

\subsection{Font Size}

\subsection{Font Shapes}

\subsection{Text Color}

\section{Paragraphs and Positioning}

\section{Verbatim Mode}