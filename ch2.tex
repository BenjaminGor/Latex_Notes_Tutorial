\chapter{Formatting of Text and Paragraphs}

\paragraph{Introduction} This chapter explains how to adjust the various aspects of text, such as fonts, shape/size/style, and positioning.

\section{About Fonts}

\subsection{The Three Font Family Types}

\paragraph{(Sans) Serif, Typewriter} In any \LaTeX{} document, the text can be typed in three different \textit{font families}: \textit{serif}, \textit{sans serif}, and \textit{typewriter}. In this book, headings (of chapters, sections, etc.) are in the sans serif family, while the remaining main text is in serif. Table \ref{tab:fontfamily} below demonstrates how to select a specific font family for a piece of text.
\begin{table}
\begin{tabularx}{\textwidth}{|l|X|l|l|}
\hline
Font Family & Command & Switch & Output \\
\hline
Serif & \texttt{\textbackslash textrm\{Hello World!\}}& \texttt{\textbackslash rmfamily} & \textrm{Hello World!} \\
\hline
Sans Serif & \texttt{\textbackslash textsf\{Hello World!\}}& \texttt{\textbackslash sffamily} & \textsf{Hello World!} \\
\hline
Typewriter & \texttt{\textbackslash texttt\{Hello World!\}}& \texttt{\textbackslash ttfamily} & \texttt{Hello World!} \\
\hline
\end{tabularx}
\caption{The commands for switching between the three font families and how they appear.}
\label{tab:fontfamily}
\end{table}
For instance, both
\begin{lstlisting}
... 
produces the following output: \par
\textsf{\lipsum[3]} % or {\sffamily \lipsum[3]}, remember the curly brackets {} to limit the scope of the \sffamily command.
\end{lstlisting}
produces the following output: \par
{\sffamily \lipsum[3]} \par
The \% symbol indicates a trailing \textit{comment} that is neither interpreted nor displayed.

\subsection{Changing the Actual Font for a Font Family}

\paragraph{Font Libraries}
Each of the previous font families is internally assigned a specific \textit{font}. To change the actual fonts, we can call the corresponding font package(s). The \textbf{\LaTeX{} Font Catalogue} \href{https://tug.org/FontCatalogue/}{https://tug.org/FontCatalogue/} provides a comprehensive list of available fonts and the way to import them. This book has substituted the \textbf{Noto Sans} font for the sans serif family, via the preamble
\begin{lstlisting}
\usepackage[T1]{fontenc}
\usepackage[sf]{noto}
\end{lstlisting}

\begin{exercisebox}
\begin{Exercise}
Change the font family just for the dummy Lipsum paragraph above to typewriter.
\end{Exercise}
\begin{Exercise}
Choose a font of your liking from the Font Catalogue to replace the original one in the book.
\end{Exercise}
\end{exercisebox}

\section{Text Attributes}

\subsection{Font Size}

\paragraph{Size Commands}
In Section \ref{sec:komaopt} we talked about setting the base global font size by \texttt{\textbackslash KOMAoptions}. However, to control the \textit{local} font size for some places, we can use the \textit{size commands}, listed in Table \ref{tab:fontsize} below.
\begin{table}[ht]
\begin{captionbeside}[test]{The various commands for text size.\footnotemark}[l][\textwidth]{
\adjustbox{valign=t}{\begin{tabularx}{0.6\textwidth}{|>{\rule{0pt}{20pt}}l|X|}
\hline
Command & Output \\
\hline
\texttt{\textbackslash tiny} & {\tiny Who am I?} \\
\hline
\texttt{\textbackslash scriptsize} & {\scriptsize Who am I?} \\
\hline
\texttt{\textbackslash footnotesize	} & {\footnotesize Who am I?} \\
\hline
\texttt{\textbackslash small} & {\small Who am I?} \\
\hline
\texttt{\textbackslash normalsize} & {\normalsize Who am I?} \\
\hline
\texttt{\textbackslash large} & {\large Who am I?} \\
\hline
\texttt{\textbackslash Large} & {\Large Who am I?} \\
\hline
\texttt{\textbackslash LARGE} & {\LARGE Who am I?} \\
\hline
\texttt{\textbackslash huge} & {\huge Who am I?} \\
\hline
\texttt{\textbackslash Huge} & {\Huge Who am I?} \\
\hline
\end{tabularx}}}
\end{captionbeside}
\label{tab:fontsize}
\end{table}
For example, writing
\begin{lstlisting}
... produces \par
{\small Though she be but little} {\LARGE she is fierce} \\ % scope
\scriptsize % switch
taken from Shakespeare's A Midsummer Night's Dream
\normalsize % back to default ...
\end{lstlisting}
\footnotetext{\texttt{\textbackslash huge} and \texttt{\textbackslash Huge} have the same size when the font size is 12 pt (but different for 10 or 11 pt).}
produces \par
{\small Though she be but little} {\LARGE she is fierce} \\
\scriptsize
taken from Shakespeare's A Midsummer Night's Dream
\normalsize
\par The \texttt{\textbackslash \textbackslash} sign breaks the current line and starts a new line right below. And again, the curly brackets \verb|{}| limit the effect of command(s) within the scope.

\paragraph{Selectfont} It is also possible to fix a numerical value for the font size using \texttt{\textbackslash fontsize\{<font\_size>\}\{<line\_spacing>\}} and \texttt{\textbackslash selectfont}. As an illustration, the code
\begin{lstlisting}
... leads to \par
{\fontsize{15pt}{21pt}\selectfont May those who accept their fate be granted happiness. May those who defy their fate be granted glory. \\
-- Princess Tutu\par} % the \par is needed to update the line spacing
\end{lstlisting}
leads to \par
{\fontsize{15pt}{21pt}\selectfont May those who accept their fate be granted happiness. May those who defy their fate be granted glory. \\
-- Princess Tutu\par}

\subsection{Font Shapes}

\paragraph{Italic, Bold, and More}
Similar to font families, there are different \textit{font shape/\allowbreak styles} such as the commonly seen italic or bold. Table \ref{tab:fontshape} above shows the relevant commands to invoke them.
\begin{table}
\begin{tabularx}{\textwidth}{|l|X|l|l|}
\hline
Font Style & Command & Switch & Output \\
\hline
Bold & \texttt{\textbackslash textbf\{"10 Downing"\}}& \texttt{\textbackslash bfseries} & \textbf{"10 Downing"} \\
\hline
Medium & \texttt{\textbackslash textmd\{"10 Downing"\}}& \texttt{\textbackslash mdseries} & \textmd{"10 Downing"} \\
\hline
Italic & \texttt{\textbackslash textit\{"10 Downing"\}}& \texttt{\textbackslash itshape} & \textit{"10 Downing"} \\
\hline
Slanted & \texttt{\textbackslash textsl\{"10 Downing"\}}& \texttt{\textbackslash slshape} & \textsl{"10 Downing"} \\
\hline
Small Caps & \texttt{\textbackslash textsc\{"10 Downing"\}}& \texttt{\textbackslash scshape} & \textsc{"10 Downing"} \\
\hline
Upright & \texttt{\textbackslash textup\{"10 Downing"\}}& \texttt{\textbackslash upshape} & \textup{"10 Downing"} \\
\hline
\end{tabularx}
\caption{The commands for different font styles. The medium/upright style is effectively the default normal.}
\label{tab:fontshape}
\end{table}
Adding to the previous example, we can write
\begin{lstlisting}
... which produces \par
\textit{\small Though she be but little} {\LARGE \bfseries \scshape she is fierce} \\ % scope
\scriptsize % switch
taken from \slshape \underline{Shakespeare's A Midsummer Night's Dream}
\normalsize \upshape % back to default ...
\end{lstlisting}
which produces \par
\textit{\small Though she be but little} {\LARGE \bfseries \scshape she is fierce} \\
\scriptsize
taken from \slshape \underline{Shakespeare's A Midsummer Night's Dream}
\normalsize \upshape \par
We also have \texttt{\textbackslash underline} and \texttt{\textbackslash emph}. You may want to try them out.

\subsection{Text Color}

\paragraph{xcolor}
While there are default colors in the \LaTeX{} system, we can load a variety of additional colors from the \verb|xcolor| package, often with flags as
\begin{lstlisting}
\usepackage[svgnames, dvipsnames]{xcolor}    
\end{lstlisting}
The reference color list can be found in \href{https://www.overleaf.com/learn/latex/Using_colors_in_LaTeX}{https://www.overleaf.com/learn/latex/\allowbreak Using\_colors\_in\_LaTeX}. To set the color for a piece of text, we can enclose it with the \texttt{\textbackslash textcolor\{<color\_name>\}\{<text>\}} command. It is also possible to change the color within a group by \texttt{\textbackslash color\{<color\_name>\}}. For instance,
\begin{lstlisting}
... outputs \par
\textcolor{Red}{Roses are red,} \\
\textcolor{Blue}{violets are blue,} \\ 
{\color{Purple} sugar is sweet and so are you.} % remember to limit the scope by the curly brackets!
\end{lstlisting}
outputs \par
\textcolor{Red}{Roses are red,} \\
\textcolor{Blue}{violets are blue,} \\ 
{\color{Purple} sugar is sweet and so are you.}

\paragraph{Self-defined colors}
It is also possible to design a custom color by the command \texttt{\textbackslash definecolor\{<color\_name>\}\{<color\_model>\}\{<values>\}}. There are $4$ possible color models: \verb|rgb|, \verb|RGB|, \verb|cmyk|, and \verb|gray|. For example,
\begin{lstlisting}
...
\definecolor{mint}{rgb}{0.24, 0.71, 0.54} % in the preamble
... gives
\textcolor{mint}{Mint Tears}
\end{lstlisting}
gives \textcolor{mint}{Mint Tears}. Color codes can be checked via \href{https://latexcolor.com/}{https://latexcolor.com/}.\par

In addition, we can mix colors by the expression \texttt{<color\_1>!<mix\_ratio>!\allowbreak <color\_2>}. For instance,
\begin{lstlisting}
\textcolor{Blue!40!Green}{Copper (II)} \textcolor{Black!50}{Sulphate}
\end{lstlisting}
is displayed as \textcolor{Blue!40!Green}{Copper (II)} \textcolor{Black!50}{Sulphate}.

\section{Paragraphs and Positioning}

\subsection{Paragraphs and Line Breaks}

\paragraph{New Lines} As explained before, the \texttt{\textbackslash\textbackslash} symbol issues a \textit{line break}, and the \texttt{\textbackslash par} command ends a paragraph and starts a new one. \\
Both of them initiate a \textit{new line}, but with (without) an extra \textit{line skip/spacing} for \texttt{\textbackslash par} (\texttt{\textbackslash\textbackslash}). There is also \texttt{\textbackslash newline} which is seldom used.
    
A blank line in the \TeX{} file has the same effect as \texttt{\textbackslash par}. They in fact end the so-called \textit{horizontal mode} and distribute the text into lines placed on the current vertical list (see \href{https://tex.stackexchange.com/questions/82664/when-to-use-par-and-when-newline-or-blank-lines}{\TeX{} StackExchange 82664}). \par 
The effects of \texttt{\textbackslash\textbackslash}, \texttt{\textbackslash par}, and blank lines can be observed right in this subsection, which is typed as
\begin{lstlisting}
\paragraph{New Lines} As explained before, ... ends a paragraph and starts a new one. \\
Both of them initiate a \textit{new line}, ... which is seldom used.
                % Here is a blank line plus this comment only
A blank line in the \TeX{} file ... placed on the current vertical list (see ...). \par
The effects of \texttt{\textbackslash\textbackslash}, \texttt{\textbackslash par}, and blank lines can be observed right in this subsection, which is typed as
... % this code block
\end{lstlisting}

\subsection{Justification and Indents}

\subsection{Length Units}

\subsection{Vertical and Horizontal Spaces}

\subsection{Boxes and Rules}

\begin{exercisebox}
\begin{Exercise}
Copy your favorite quote or paragraph to the document, and use the commands/techniques introduced in these two sections to make it beautiful and stylish.    
\end{Exercise}
\end{exercisebox}

\section{Verbatim Mode}